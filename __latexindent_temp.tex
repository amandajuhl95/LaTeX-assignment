\documentclass{article}
\usepackage{graphicx}
\usepackage{amssymb}
\usepackage{array}
\usepackage{tabularx}
\usepackage{biblatex}
\usepackage{caption}
\usepackage{subcaption}
\usepackage{placeins}
\usepackage{todonotes}
\addbibresource{Bibliography/latex-opgave.bib}

\author{Amanda Juhl Hansen}
\title{Assignment 2 - preperation for bachelor thesis}

\begin{document}

\clearpage\maketitle
\thispagestyle{empty}
\tableofcontents
\listoffigures
\listoftables

\newpage
    \section{Task 1}
    
    Find requirements of bachelor thesis. Write a LATEX document explaining your
    findings. Document your sources.
    \newline 

    Da besvarelsen skal være brugbar for os som studerende,har vi valgt at tage udgangspunkt i CPH Business krav til vores kommende bachelor projekt. \cite{softwebsite}

    \subsection*{Krav til Rapporten}
    \begin{itemize}
        \item Da en bachelor svarer til 15 ECTS points, forventes det at den studerende bruger 412,5 time på bachelor projektet.  
            \[ 15 * 27,4 timer = 412,5 timer \]
        \item Det maksimale antal sider afhænger af gruppens størrelse og udregnes således: \(maksantalsider = 40 + 20 * antalstuderende\)
        \item Hvis rapporten er under \(2/3\) af det maksimale antal sider, anses den for at være kort. Der er dog ikke et officelt minimumskrav.  
        \item Rapporten skal skrives på enten dansk eller engelsk.
        \item Rapporten skal indeholde en gennemgående beskrivelse af det udarbejdede projekt, samt evaluering og refleksion over dette.
    \end{itemize}
        \subsection*{Rapportens indhold}
        \begin{enumerate}
            \item Forside (titel, studerendes navne og id, skole og vejleder)
            \item Abstract (ca. 4 linjer og skrives til aller sidst)
            \item Indholdsfortegnelse 
            \item Introduktion
            \begin{itemize}
                \item Motivation
                \item Forventede resultat
                \item Opgaver for at kunne opnå det forventede resultat
                \item Scope
                \item Kort beskrivelse af hvert afsnit efterfølgende
            \end{itemize}
            \item Undersøgelse og redegørelse af teknologier og teori
            \item Krav specfikationer og design
            \item Implementation og udvikling 
            \item Konklusion
            \item Littearturliste og bilag 
            \item Præsentation (eventuelt powerpoint)
        \end{enumerate}
    
      

\newpage
    \section{Task 2}

    Produce a template (in LATEX, of course) that you can use in your bachelor
    thesis. It should be rich with examples of the following (ie. one of each):

        \subsection{Danish letters}
            æ ø å

        \subsection{Graphics}

                \begin{figure}[h]
                    \caption{Caption above image}
                    \includegraphics[width=0.5\linewidth]{Images/cat.jpg}
                    \centering
                    \label{fig:cat1}
                \end{figure}
                \FloatBarrier

                \begin{figure}[h]
                    \begin{subfigure}{0.45\linewidth}
                        \centering
                        \includegraphics[width=\textwidth]{Images/tireddog.jpg}
                        \caption{This is a tired dog}
                        \label{fig:figure1}
                    \end{subfigure}
                \hspace{0.5cm}
                    \begin{subfigure}{0.45\linewidth}
                        \centering
                        \includegraphics[width=\textwidth]{Images/happydog.jpg}
                        \caption{This is a happy dog}
                        \label{fig:figure2}
                    \end{subfigure}
                \caption{This is two doggies}
                \label{fig:thebigpicture}
                \end{figure}
                \FloatBarrier

        \subsection{Reference}

        Figure \ref{fig:cat1} on page \pageref{fig:cat1} shows a picture of a cat.
    
        \newpage
        \subsection{Paragraphs}
                \paragraph{Paragraph}
                This is an example of a paragraph
                    \subparagraph{Subparagraph}
                    This is an example of a subparagraph
        
        \newpage
        \subsection{Lists}
            \subsubsection{Bullet points and enumerate}
            \begin{enumerate}
                \item The labels consists of sequential numbers.
                \begin{itemize}
                    \item This is a bullet point.
                    \item This is another bullet point.
                    \begin{itemize}
                        \item This is the second level
                    \end{itemize}
                \end{itemize}
                \item The numbers starts at 1 with every call to the enumerate environment.
                \begin{enumerate}
                    \item Second level of this list
                    \item Second level
                \end{enumerate}
            \end{enumerate}
                

            \subsubsection{Alternative bullet points}
            \begin{itemize}
                \item[$\square$]  Custom made bullet point
            \end{itemize}

        \newpage
        \subsection{Tables}
              \subsubsection{Various horizontal alignments in columns (left, right, centered)}
        
        \begin{table}[h!]
           
            \begin{tabularx}{0.8\textwidth} { 
                | >{\raggedright\arraybackslash}X 
                | >{\centering\arraybackslash}X 
                | >{\raggedleft\arraybackslash}X | }
                \hline
                one & two & three \\ 
                \hline
                four & five & six \\ 
                \hline
                seven & eight & nine \\ 
                \hline
                \end{tabularx}
            \centering
            \caption{This is a description about the table}
            \label{table:numbers}
        \end{table}

            \subsubsection{Cell spanning multiple columns}
            
            \begin{center}
            \begin{tabular}{ |c | c| }
                \hline 
                \multicolumn{2}{|c|}{Name Dropping} \\ [1ex]
                \hline 
                Frontname & Year of birth \\ [1ex]
                \hline
                Lisa &  1997 \\
                Kenny &  1943 \\
                Else Pelse & 2015 \\
                Ani & 2000 \\
                Verner & 1982 \\ [1ex]
                \hline
               \end{tabular}
            \end{center}

            \subsubsection{Reference}
            Table \ref{table:numbers} is on page \pageref{table:numbers}
        
        \newpage
        \subsection{Code listing - emphasized key words}
        Java is a popular programming language, created in 1995. It is owned by \emph{Oracle}, and more than \textbf{3 billion} devices run Java. It is \emph{open-source}, furthermore it's easy to learn and simple to use.
        
        \newpage
        \subsection{Math equations}

        \subsubsection{Inline equations (in text)}
        You can only use pythagoras, \(a^2 + b^2 = c^2\) for calculatings on 90 degree triangles. To find area of a triangle you use this formula: \begin{math} \frac{b * h}{2} \end{math}. And last but not least, finding the volume of a box you use this formula: $V = l * b * h $. 

        \subsubsection{Display equations (on separate line)}
        Examples of fractions, summations, products, roots, powers
        
        This is an example of a fraction: 
        \[\frac{1}{2}\]
        
        This is an example of a summation: 
        \[\sum_{n=2}^{\infty} 2^{-n} = 1 \]

        This is an example of products:
        \[ \prod_{i=a}^{b} f(i) \]

        This is an example of roots:
        \[ \sqrt{25} \]

        This is an example of powers:
        \[ x^2 \]

        \subsection{Todo notes}
        \todo{This is a todo note. }

        \newpage
        \subsection{Bibliography of book, article and internet link}
        
        Let's cite! \cite{Clear} 
        Let's cite! \cite{latexcompanion}
        Let's cite! \cite{knuthwebsite}
        \printbibliography


\end{document}